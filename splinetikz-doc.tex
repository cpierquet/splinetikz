% !TeX TXS-program:compile = txs:///pdflatex/[--shell-escape]

\documentclass{article}
\usepackage[utf8]{inputenc}
\usepackage[T1]{fontenc}
\usepackage[upright]{fourier}
\usepackage[scaled=0.875]{helvet}
\renewcommand\ttdefault{lmtt}
\usepackage[scaled=0.875]{cabin}
\usepackage{splinetikz}
\usepackage[french]{babel}
\usepackage{siunitx}
\usepackage{graphics}
\usepackage{hvlogos}
\usepackage{simplekv}
\usepackage{listofitems}
\usepackage{xintexpr}
\usepackage{codehigh}
\usepackage{minted}
\sisetup{locale=FR}
\usepackage{geometry}
\geometry{margin=1.5cm}
\usepackage{newverbs}
\newverbcommand{\pverb}{\color{purple}}{}
\newverbcommand{\rverb}{\color{red}}{}
\newverbcommand{\vverb}{\color{ForestGreen}}{}
\newverbcommand{\averb}{\color{Aquamarine}}{}
\newverbcommand{\overb}{\color{orange}}{}
\setlength{\parindent}{0pt}
\definecolor{LightGray}{gray}{0.9}

\begin{document}

%gestion de la fenêtre v2 directement dans le tikzpicture
\tikzset{%
	xmin/.store in=\xmin,xmin/.default=-5,xmin=-5,
	xmax/.store in=\xmax,xmax/.default=5,xmax=5,
	ymin/.store in=\ymin,ymin/.default=-5,ymin=-5,
	ymax/.store in=\ymax,ymax/.default=5,ymax=5,
	xgrille/.store in=\xgrille,xgrille/.default=1,xgrille=1,
	xgrilles/.store in=\xgrilles,xgrilles/.default=0.5,xgrilles=0.5,
	ygrille/.store in=\ygrille,ygrille/.default=1,ygrille=1,
	ygrilles/.store in=\ygrilles,ygrilles/.default=0.5,ygrilles=0.5,
	xunit/.store in=\xunit,unit/.default=1,xunit=1,
	yunit/.store in=\yunit,unit/.default=1,yunit=1
}
\newcommand\tgrilles[1][ultra thin,lightgray]{%
	\draw[xstep=\xgrilles,ystep=\ygrilles,#1] (\xmin,\ymin) grid (\xmax,\ymax);%
}
\newcommand\tgrillep[1][thin,gray]{%
	\draw[xstep=\xgrille,ystep=\ygrille,#1] (\xmin,\ymin) grid (\xmax,\ymax);%
}

\newcommand\genfenetre{%
	%styles
	\tikzset{noeudexpl/.style={purple,font=\sffamily\small}}
	\tikzset{portionexpl/.style={orange,thick,<->}}
	\tikzset{expl/.style={midway,inner sep=1pt,above right=0,orange,font=\sffamily\scriptsize,rotate=45}}
	\tikzset{coeffs/.style={Aquamarine!50!black,circle,draw=Aquamarine,thick,fill=Aquamarine!5,font=\small\ttfamily}}
	\tikzset{tangente/.style={teal,line width=1pt,dashed}}
	%grilles & axes
	\tgrilles[line width=0.3pt,lightgray!50]
	\tgrillep[line width=0.6pt,lightgray!50]
	\draw[line width=1.5pt,->,gray] (\xmin,0)--(\xmax,0) ;
	\draw[line width=1.5pt,->,gray] (0,\ymin)--(0,\ymax) ;
	\foreach \x in {0,1,...,10} {\draw[gray,line width=1.5pt] (\x,4pt) -- (\x,-4pt) ;}
	\foreach \y in {0,1,...,6} {\draw[gray,line width=1.5pt] (4pt,\y) -- (-4pt,\y) ;}
}

\newcommand\gennotice{%
	%notice
	\draw (0,1) node[noeudexpl,below] {point 1} ;
	\draw (4,3.667) node[noeudexpl,above] {point 2} ;
	\draw (7.5,1.75) node[noeudexpl,below] {point 3} ;
	\draw (9,2) node[noeudexpl,above] {point 4} ;
	\draw (10,0) node[noeudexpl,below] {point 5} ;
	\draw[portionexpl] (0,6)--(4,6) node[expl] {portion 1} ;
	\draw[portionexpl] (4,6)--(7.5,6) node[expl] {portion 2} ;
	\draw[portionexpl] (7.5,6)--(9,6) node[expl] {portion 3} ;
	\draw[portionexpl] (9,6)--(10,6) node[expl] {portion 4} ;
	\draw[orange,densely dashed,thick] (4,0)--(4,6) (7.5,0)--(7.5,6) (9,0)--(9,6) (10,0)--(10,6) ;
	%tangentes
	\draw[tangente] (0,1)--(1,1) ;
	\draw[tangente,domain=3:5] plot (\x,{-1/3*(\x-9)+2}) ;
	\draw[tangente] (6.5,1.75)--(8.5,1.75) ;
	\draw[tangente,domain=8:10] plot (\x,{-1/3*(\x-9)+2}) ;
	\draw[tangente,domain=9.5:10] plot (\x,{-10*(\x-10)+0}) ;%
}

\newcommand\listecoeffs[4]{%
	\draw (0,5.5) node[left,Aquamarine,font=\small\ttfamily] {Coeffs} ;
	\node[coeffs] at (2,5.5) {#1} ;
	\node[coeffs] at ({(4+7.5)/2},5.5) {#2} ;
	\node[coeffs] at ({(7.5+9)/2},5.5) {#3} ;
	\node[coeffs] at ({(9+10)/2},5.5) {#4} ;%
}

\part*{Courbe d'interpolation}

\section{Introduction}

\subsection{Outils disponibles}

Les \rverb|splines cubiques| sont utiles pour \og tracer \fg{} des portions de courbes, régulières entre deux points donnés, avec gestion des tangentes de sortie et d'entrée.

On peut les déterminer, par résolution de systèmes, donc grâce à un outil externe.

\smallskip

\TikZ{} permet, grâce à l'outil \overb|..controls| (lié aux courbes de Bézier) d'obtenir un résultat très proche de l'interpolation cubique, et ce avec un \textit{paramètre} qui vaut \overb|3|.

\smallskip

Il existe également l'option \vverb|in=...,out=...| pour gérer les angles de sortie et d'entrée de l'option chemin \vverb|to|.


\subsection{Comparaison des outils}

{\scriptsize \begin{codehigh}[language=latex/latex2,style/main=red!25,style/code=red!25]
...
\tikzset{courbe/.style={line width=1.25pt,red,samples=250}}
\draw[courbe,domain=-5:-3] plot(\x,{-3/4*\x*\x*\x+-9*\x*\x+-135/4*\x+-79/2}) ;
\draw[courbe,domain=-3:-1] plot(\x,{1/2*\x*\x*\x+3*\x*\x+9/2*\x+1}) ;
\draw[courbe,domain=-1:2] plot(\x,{-0.185185*\x*\x*\x+0.27777*\x*\x+1.111111*\x+-0.351851}) ;
...
\end{codehigh}}

\begin{center}
	\begin{tikzpicture}[x=0.6cm,y=0.6cm,xmin=-6,xmax=3,xgrille=1,xgrilles=0.5,ymin=-3,ymax=3,ygrille=1,ygrilles=0.5]
		\tikzset{courbe/.style={line width=1.25pt,red,samples=250}}
		%grilles & axes
		\tgrilles[line width=0.3pt,lightgray!50]
		\tgrillep[line width=0.6pt,lightgray!50]
		\draw[line width=1.5pt,->,gray] (\xmin,0)--(\xmax,0) ;
		\draw[line width=1.5pt,->,gray] (0,\ymin)--(0,\ymax) ;
		\foreach \x in {-5,-4,...,2} {\draw[gray,line width=1.5pt] (\x,4pt) -- (\x,-4pt) ;}
		\foreach \y in {-2,-1,...,2} {\draw[gray,line width=1.5pt] (4pt,\y) -- (-4pt,\y) ;}
		\foreach \Point in {(-5,-2),(-5,-2),(-3,1),(-1,-1),(2,1.5)}
			\filldraw[red] \Point circle[radius=2pt] ;
		%\clip (\xmin,\ymin) rectangle (\xmax,\ymax) ; %on restreint les fonctions à la fenêtre
		%les splines en pgf
		\draw[courbe,domain=-5:-3] plot(\x,{-3/4*\x*\x*\x+-9*\x*\x+-135/4*\x+-79/2}) ;
		\draw[courbe,domain=-3:-1] plot(\x,{1/2*\x*\x*\x+3*\x*\x+9/2*\x+1}) ;
		\draw[courbe,domain=-1:2] plot(\x,{-0.185185185185185*\x*\x*\x+0.277777777777778*\x*\x+1.11111111111111*\x+-0.351851851851852}) ;
	\end{tikzpicture}
\end{center}

\medskip

{\scriptsize \begin{codehigh}[language=latex/latex2,style/main=orange!25,style/code=orange!25]
...
\tikzset{courbe/.style={line width=1.25pt,red,samples=250}}
\draw[orange,line width=1.25pt] (-5,-2) ..controls +(0:0.67) and +(180:0.67).. (-3,1) ;
\draw[orange,line width=1.25pt] (-3,1) ..controls +(0:0.67) and +(180:0.67).. (-1,-1) ;
\draw[orange,line width=1.25pt] (-1,-1) ..controls +(0:1) and +(180:1).. (2,1.5) ;
...
\end{codehigh}}

\begin{center}
	\begin{tikzpicture}[x=0.6cm,y=0.6cm,xmin=-6,xmax=3,xgrille=1,xgrilles=0.5,ymin=-3,ymax=3,ygrille=1,ygrilles=0.5]
		\tikzset{courbe/.style={line width=1.25pt,red,samples=250}}
		%grilles & axes
		\tgrilles[line width=0.3pt,lightgray!50]
		\tgrillep[line width=0.6pt,lightgray!50]
		\draw[line width=1.5pt,->,gray] (\xmin,0)--(\xmax,0) ;
		\draw[line width=1.5pt,->,gray] (0,\ymin)--(0,\ymax) ;
		\foreach \x in {-5,-4,...,2} {\draw[gray,line width=1.5pt] (\x,4pt) -- (\x,-4pt) ;}
		\foreach \y in {-2,-1,...,2} {\draw[gray,line width=1.5pt] (4pt,\y) -- (-4pt,\y) ;}
		\foreach \Point in {(-5,-2),(-5,-2),(-3,1),(-1,-1),(2,1.5)}
			\filldraw[orange] \Point circle[radius=2pt] ;
		%\clip (\xmin,\ymin) rectangle (\xmax,\ymax) ; %on restreint les fonctions à la fenêtre
		%controls
		\draw[orange,line width=1.25pt] (-5,-2) ..controls +(0:0.67) and +(180:0.67).. (-3,1) ;
		\draw[orange,line width=1.25pt] (-3,1) ..controls +(0:0.67) and +(180:0.67).. (-1,-1) ;
		\draw[orange,line width=1.25pt] (-1,-1) ..controls +(0:1) and +(180:1).. (2,1.5) ;
	\end{tikzpicture}
\end{center}

\medskip

{\scriptsize \begin{codehigh}[language=latex,style/main=ForestGreen!25,style/code=ForestGreen!25]
...
\tikzset{courbe/.style={line width=1.25pt,red,samples=250}}
\draw[ForestGreen,line width=1.25pt] (-5,-2) to[out=0,in=180] (-3,1) ;
\draw[ForestGreen,line width=1.25pt] (-3,1) to[out=0,in=180] (-1,-1) ;
\draw[ForestGreen,line width=1.25pt] (-1,-1) to[out=0,in=180] (2,1.5) ;
...
\end{codehigh}}

\begin{center}
	\begin{tikzpicture}[x=0.6cm,y=0.6cm,xmin=-6,xmax=3,xgrille=1,xgrilles=0.5,ymin=-3,ymax=3,ygrille=1,ygrilles=0.5]
		\tikzset{courbe/.style={line width=1.25pt,red,samples=250}}
		%grilles & axes
		\tgrilles[line width=0.3pt,lightgray!50]
		\tgrillep[line width=0.6pt,lightgray!50]
		\draw[line width=1.5pt,->,gray] (\xmin,0)--(\xmax,0) ;
		\draw[line width=1.5pt,->,gray] (0,\ymin)--(0,\ymax) ;
		\foreach \x in {-5,-4,...,2} {\draw[gray,line width=1.5pt] (\x,4pt) -- (\x,-4pt) ;}
		\foreach \y in {-2,-1,...,2} {\draw[gray,line width=1.5pt] (4pt,\y) -- (-4pt,\y) ;}
		\foreach \Point in {(-5,-2),(-5,-2),(-3,1),(-1,-1),(2,1.5)}
			\filldraw[ForestGreen] \Point circle[radius=2pt] ;
		%\clip (\xmin,\ymin) rectangle (\xmax,\ymax) ; %on restreint les fonctions à la fenêtre
		%controls
		\draw[ForestGreen,line width=1.25pt] (-5,-2) to[out=0,in=180] (-3,1) ;
		\draw[ForestGreen,line width=1.25pt] (-3,1) to[out=0,in=180] (-1,-1) ;
		\draw[ForestGreen,line width=1.25pt] (-1,-1) to[out=0,in=180] (2,1.5) ;
	\end{tikzpicture}
\end{center}

\newpage

\section{L'outil \og splinetikz \fg}

\subsection{Explications et définitions}

On va utiliser les notions suivantes pour paramétrer le tracé \og automatique \fg{} par \verb|..controls| :
%
\begin{itemize}
	\item il faut rentrer les \textcolor{purple}{\textsf{points de contrôle}} ;
	\item il faut préciser les \textcolor{ForestGreen}{\textsf{pentes des tangentes}} (pour le moment on travaille avec les mêmes à gauche et à droite\ldots) ;
	\item on peut paramétrer les \textcolor{Aquamarine}{\textsf{coefficients}} pour \og affiner \fg{} les portions.
\end{itemize}

\medskip

Pour déclarer les paramètres :
%
\begin{itemize}
	\item liste des points de contrôle par : \verb|liste=x1/y1/d1§x2/y2/d2§...|
	\begin{itemize}
		\item il faut au-moins deux points ;
		\item avec les points \pverb|(xi;yi)| et \vverb|f'(xi)=di|.
	\end{itemize}
	\item coefficients de contrôle par \verb|coeffs=...| :
	\begin{itemize}
		\item \averb|coeffs=x| pour mettre tous les coefficients à x ;
		\item \averb|coeffs=C1§C2§...| pour spécifier les coefficients par portion (donc il faut avoir autant de § que pour les points !) ;
		\item \averb|coeffs=C1G/C1D§...| pour spécifier les coefficients par portion et par partie gauche/droite ;
		\item on peut mixer avec \averb|coeffs=C1§C2G/C2D§...|.
	\end{itemize}
\end{itemize}

\subsection{Clés et options}

La commande \verb|\splinetikz| se présente sous la forme :

\begin{minted}[frame=lines,framesep=2mm,bgcolor=LightGray,fontsize=\footnotesize,tabsize=2]{tex}
\begin{tikzpicture}
	...
	\splinetikz[liste=...,coeffs=...,affpoints=...,couleur=...,epaisseur=...,taillepoints=...,couleurpoints=...,style=...]
	...
\end{tikzpicture}
\end{minted}

Certains paramètres peuvent être gérés directement dans la commande \verb|\splinetikz| :

\begin{itemize}
	\item la couleur de la courbe est \textcolor{red}{rouge}, gérée par la \textsf{clé} \texttt{<couleur=...>} ;
	\item l'épaisseur de la courbe est de \textcolor{red}{1.25pt}, gérée par la \textsf{clé} \texttt{<epaisseur=...>} ;
	\item du style supplémentaire pour la courbe peut être rajouté, grâce à la \textsf{clé} \texttt{<style=...>} ;
	\item les coefficients de \textit{compensation} sont par faut à 3, gérés par la \textsf{clé} \texttt{<coeffs=...>}
	\item les points de contrôle ne sont pas affichés par défaut, mais \textsf{clé booléenne} \texttt{<affpoints=true>} permet de les afficher ;
	\item la taille des points de contrôle est géré par la \textsf{clé} \texttt{<taillepoints=...>}.
\end{itemize}

\medskip

\begin{center}
	\begin{tikzpicture}[x=0.9cm,y=0.9cm,xmin=-1,xmax=11,xgrille=1,xgrilles=0.5,ymin=-1,ymax=7,ygrille=1,ygrilles=0.5]
		\genfenetre
		\splinetikz[liste=0/1/0§4/3.667/-0.333§7.5/1.75/0§9/2/-0.333§10/0/-10,affpoints=true,coeffs=3,couleur=red]
		\gennotice
		\listecoeffs{3}{3}{3}{3}
	\end{tikzpicture}
\end{center}

\hspace{3cm}\verb|\splinetikz[%|

\hspace{4cm}\verb|liste=|%
\pverb|0/1|\vverb|/0|\verb|§|\pverb|4/3.667|\vverb|/-0.333|\verb|§|\pverb|7.5/1.75|\vverb|/0|\verb|§|\pverb|9/2|\vverb|/-0.333|\verb|§|\pverb|10/0|\vverb|/-10|\verb|,%|

\hspace{4cm}\verb|coeffs=|\averb|3|\verb|,%|

\hspace{4cm}\verb|affpoints=true|\verb|,%|

\hspace{4cm}\verb|couleur=red|\verb|]|

\newpage

\section{Influence(s) des coefficients de \textit{compensation}}

\subsection{Avec une légère modification pour la dernière portion}

\begin{center}
	\begin{tikzpicture}[x=1cm,y=1cm,xmin=-1,xmax=11,xgrille=1,xgrilles=0.5,ymin=-1,ymax=7,ygrille=1,ygrilles=0.5]
		\genfenetre
		\splinetikz[liste=0/1/0§4/3.667/-0.333§7.5/1.75/0§9/2/-0.333§10/0/-10,affpoints=true,coeffs=3§3§3§2]
		\gennotice
		\listecoeffs{3}{3}{3}{2}
	\end{tikzpicture}
\end{center}

\hspace{3cm}\verb|\splinetikz[%|

\hspace{4cm}\verb|liste=|%
	\pverb|0/1|\vverb|/0|\verb|§|\pverb|4/3.667|\vverb|/-0.333|\verb|§|\pverb|7.5/1.75|\vverb|/0|\verb|§|\pverb|9/2|\vverb|/-0.333|\verb|§|\pverb|10/0|\vverb|/-10|\verb|,%|
	
\hspace{4cm}\verb|coeffs=|\averb|3|\verb|§|\averb|3|\verb|§|\averb|3|\verb|§|\averb|2|\verb|,%|

\hspace{4cm}\verb|affpoints=true|\verb|]|

\subsection{Avec une gestion plus fine de la dernière partie}

\begin{center}
	\begin{tikzpicture}[x=1cm,y=1cm,xmin=-1,xmax=11,xgrille=1,xgrilles=0.5,ymin=-1,ymax=7,ygrille=1,ygrilles=0.5]
		\genfenetre
		\splinetikz[liste=0/1/0§4/3.667/-0.333§7.5/1.75/0§9/2/-0.333§10/0/-10,affpoints=true,coeffs=3§3§3§2/1]
		\gennotice
		\listecoeffs{3/3}{3/3}{3/3}{2/1}
	\end{tikzpicture}
\end{center}

\hspace{3cm}\verb|\splinetikz[%|

\hspace{4cm}\verb|liste=|%
	\pverb|0/1|\vverb|/0|\verb|§|\pverb|4/3.667|\vverb|/-0.333|\verb|§|\pverb|7.5/1.75|\vverb|/0|\verb|§|\pverb|9/2|\vverb|/-0.333|\verb|§|\pverb|10/0|\vverb|/-10|\verb|,%|

\hspace{4cm}\verb|coeffs=|\averb|3/3|\verb|§|\averb|3/3|\verb|§|\averb|3/3|\verb|§|\averb|2/1|\verb|,%| ou \verb|coeffs=|\averb|3|\verb|§|\averb|3|\verb|§|\averb|3|\verb|§|\averb|2/1|\verb|,%|

\hspace{4cm}\verb|affpoints=true|\verb|]|

\newpage

\subsection{Compléments visuels sur les coefficients de\textit{compensation}}

Dans la majorité des cas, le coefficient \textcircled{3} permet d'obtenir une courbe (ou une portion) très satisfaisante !

Dans certains cas, notamment si l'une des pentes est relativement \og forte \fg{} et/ou si l'intervalle horizontal de la portion est relativement \og étroit \fg, il se peut que la portion paraisse un peu trop \og abrupte \fg{}.

On peut dans ce cas \textit{jouer} sur les coefficients de cette portion pour \textit{arrondir} un peu tout cela !

\begin{center}
	\tikzset{coeffs/.style={Aquamarine!50!black,circle,draw=Aquamarine,thick,fill=Aquamarine!5,font=\small\ttfamily}}
	\begin{tikzpicture}[x=3cm,y=3cm,xmin=0,xmax=2,xgrilles=0.25,ymin=0,ymax=2.25,ygrilles=0.25]
		\tgrilles
		\draw[line width=1.5pt,->,darkgray] (\xmin,0)--(\xmax,0) ;
		\draw[line width=1.5pt,->,darkgray] (0,\ymin)--(0,\ymax) ;
		\splinetikz[liste=0/1.5/0§1/2/-0.333§2/0/-5,affpoints=true,coeffs=3]
		\draw[line width=1.5pt,ForestGreen,dashed] (0,1.5)--(0.5,1.5) ;
		\draw[line width=1.5pt,ForestGreen,dashed,domain=0.5:1.5] plot (\x,{-1/3*(\x-1)+2}) ;
		\draw[line width=1.5pt,ForestGreen,dashed,domain=1.66:2] plot (\x,{-5*(\x-2)+0}) ;
		\draw (1,0) node[below,blue,font=\large] {\verb|coeffs=3|} ;
		%les coeffs
		\node[coeffs,below right=5pt] at (0,1.5) {3} ;
		\node[coeffs,below left=5pt] at (1,2) {3} ;
		\node[coeffs,above right=5pt] at (1,2) {3} ;
		\node[coeffs,above left=5pt] at (2,0) {3} ;
	\end{tikzpicture}
	~~
	\begin{tikzpicture}[x=3cm,y=3cm,xmin=0,xmax=2,xgrilles=0.25,ymin=0,ymax=2.25,ygrilles=0.25]
		\tgrilles
		\draw[line width=1.5pt,->,darkgray] (\xmin,0)--(\xmax,0) ;
		\draw[line width=1.5pt,->,darkgray] (0,\ymin)--(0,\ymax) ;
		\splinetikz[liste=0/1.5/0§1/2/-0.333§2/0/-5,affpoints=true,coeffs=2,couleur=red]
		\draw[line width=1.5pt,ForestGreen,dashed] (0,1.5)--(0.5,1.5) ;
		\draw[line width=1.5pt,ForestGreen,dashed,domain=0.5:1.5] plot (\x,{-1/3*(\x-1)+2}) ;
		\draw[line width=1.5pt,ForestGreen,dashed,domain=1.66:2] plot (\x,{-5*(\x-2)+0}) ;
		\draw (1,0) node[below,red,font=\large] {\verb|coeffs=2|} ;
		%les coeffs
		\node[coeffs,below right=5pt] at (0,1.5) {2} ;
		\node[coeffs,below left=5pt] at (1,2) {2} ;
		\node[coeffs,above right=5pt] at (1,2) {2} ;
		\node[coeffs,above left=5pt] at (2,0) {2} ;
	\end{tikzpicture}
	
	\smallskip
	
	\begin{tikzpicture}[x=3cm,y=3cm,xmin=0,xmax=2,xgrilles=0.25,ymin=0,ymax=2.25,ygrilles=0.25]
		\tgrilles
		\draw[line width=1.5pt,->,darkgray] (\xmin,0)--(\xmax,0) ;
		\draw[line width=1.5pt,->,darkgray] (0,\ymin)--(0,\ymax) ;
		\splinetikz[liste=0/1.5/0§1/2/-0.333§2/0/-5,affpoints=true,coeffs=3§2,couleur=purple]
		\draw[line width=1.5pt,ForestGreen,dashed] (0,1.5)--(0.5,1.5) ;
		\draw[line width=1.5pt,ForestGreen,dashed,domain=0.5:1.5] plot (\x,{-1/3*(\x-1)+2}) ;
		\draw[line width=1.5pt,ForestGreen,dashed,domain=1.66:2] plot (\x,{-5*(\x-2)+0}) ;
		\draw (1,0) node[below,purple,font=\large] {\verb|coeffs=3§2|} ;
		%les coeffs
		\node[coeffs,below right=5pt] at (0,1.5) {3} ;
		\node[coeffs,below left=5pt] at (1,2) {3} ;
		\node[coeffs,above right=5pt] at (1,2) {2} ;
		\node[coeffs,above left=5pt] at (2,0) {2} ;
	\end{tikzpicture}
	~~
	\begin{tikzpicture}[x=3cm,y=3cm,xmin=0,xmax=2,xgrilles=0.25,ymin=0,ymax=2.25,ygrilles=0.25]
		\tgrilles
		\draw[line width=1.5pt,->,darkgray] (\xmin,0)--(\xmax,0) ;
		\draw[line width=1.5pt,->,darkgray] (0,\ymin)--(0,\ymax) ;
		\splinetikz[liste=0/1.5/0§1/2/-0.333§2/0/-5,affpoints=true,coeffs=3§2/1,couleur=orange]
		\draw[line width=1.5pt,ForestGreen,dashed] (0,1.5)--(0.5,1.5) ;
		\draw[line width=1.5pt,ForestGreen,dashed,domain=0.5:1.5] plot (\x,{-1/3*(\x-1)+2}) ;
		\draw[line width=1.5pt,ForestGreen,dashed,domain=1.66:2] plot (\x,{-5*(\x-2)+0}) ;
		\draw (1,0) node[below,orange,font=\large] {\verb|coeffs=3§2/1|} ;
		%les coeffs
		\node[coeffs,below right=5pt] at (0,1.5) {3} ;
		\node[coeffs,below left=5pt] at (1,2) {3} ;
		\node[coeffs,above right=5pt] at (1,2) {2} ;
		\node[coeffs,above left=5pt] at (2,0) {1} ;
	\end{tikzpicture}
\end{center}

\begin{minted}[frame=lines,framesep=2mm,bgcolor=LightGray,fontsize=\footnotesize,tabsize=2]{tex}
\begin{tikzpicture}
	...
	%le spline
	\splinetikz[liste=0/1.5/0§1/2/-0.333§2/0/-5,affpoints=true,coeffs=...,couleur=...]
	%les tangentes
	\draw[line width=1.5pt,ForestGreen,dashed] (0,1.5)--(0.5,1.5) ;
	\draw[line width=1.5pt,ForestGreen,dashed,domain=0.5:1.5] plot (\x,{-1/3*(\x-1)+2}) ;
	\draw[line width=1.5pt,ForestGreen,dashed,domain=1.66:2] plot (\x,{-5*(\x-2)+0}) ;
	...
\end{tikzpicture}
\end{minted}

\newpage

\section{L'outil \og tangentetikz \fg{}}

\subsection{Définitions}

En parallèle de l'outil \verb|\splinetikz|, il existe l'outil \verb|\tangentetikz| qui va permettre de tracer des tangentes à l'aide de la liste de points précédemment définie pour l'outil \verb|\splinetikz|.

\smallskip

NB : il peut fonctionner indépendamment de l'outil \verb|\splinetikz| puisque la liste des points de travail est gérée de manière autonome !

\medskip

La commande \verb|\tangentetikz| se présente sous la forme :

\begin{minted}[frame=lines,framesep=2mm,bgcolor=LightGray,fontsize=\footnotesize,tabsize=2]{tex}
\begin{tikzpicture}
	...
	\tangentetikz[liste=...,couleur=...,epaisseur=...,xl=...,xr=...,style=...,point=...]
	...
\end{tikzpicture}
\end{minted}

Cela permet de tracer la tangente :
%
\begin{itemize}
	\item au point numéro numéro \texttt{<point>} de la liste \texttt{<liste>}, de coordonnées \textsf{xi/yi} avec la pente \textsf{di} ;
	\item avec une épaisseur de \texttt{<epaisseur>}, une couleur \texttt{<couleur>} et un style additionnel \texttt{<style>} ;
	\item en la traçant à partir de \texttt{<xl>} avant \textsf{xi} et jusqu'à \texttt{<xr>} après \textsf{xi}.
\end{itemize}

\subsection{Exemple}

\begin{center}
	\begin{tikzpicture}[x=3cm,y=3cm,xmin=0,xmax=2,xgrilles=0.25,ymin=0,ymax=2.25,ygrilles=0.25]
		\tikzset{noeudexpl/.style={purple,font=\sffamily\small}}
		\tgrilles
		\draw[line width=1.5pt,->,darkgray] (\xmin,0)--(\xmax,0) ;
		\draw[line width=1.5pt,->,darkgray] (0,\ymin)--(0,\ymax) ;
		\draw (0,1.5) node[noeudexpl,below] {point 1} ;
		\draw (1,2) node[noeudexpl,below] {point 2} ;
		\draw (2,0) node[noeudexpl,above left] {point 3} ;
		%spline
		\splinetikz[liste=0/1.5/0§1/2/-0.333§2/0/-5,affpoints=true,coeffs=3§2,couleur=red]
		%tangente
		\tangentetikz[liste=0/1.5/0§1/2/-0.333§2/0/-5,xl=0,xr=0.5,couleur=ForestGreen,style=dashed]
		\tangentetikz[liste=0/1.5/0§1/2/-0.333§2/0/-5,xl=0.5,xr=0.75,couleur=orange,style=dotted,point=2]
		\tangentetikz[liste=0/1.5/0§1/2/-0.333§2/0/-5,xl=0.33,xr=0,couleur=blue,style=densely dashed,point=3]
		%explications
		\draw[<->,very thick,darkgray] (0.5,2.2)--(1,2.2) node[midway,above,font=\sffamily] {xl} ;
		\draw[<->,very thick,darkgray] (1,2.2)--(1.75,2.2) node[midway,above,font=\sffamily] {xr};
	\end{tikzpicture}
\end{center}

\begin{minted}[frame=lines,framesep=2mm,bgcolor=LightGray,fontsize=\footnotesize,tabsize=2]{tex}
\begin{tikzpicture}
	...
	%spline
	\splinetikz[liste=0/1.5/0§1/2/-0.333§2/0/-5,affpoints=true,coeffs=3§2,couleur=red]
	%tangente
	\tangentetikz[liste=0/1.5/0§1/2/-0.333§2/0/-5,xl=0,xr=0.5,couleur=ForestGreen,style=dashed]
	\tangentetikz[liste=0/1.5/0§1/2/-0.333§2/0/-5,xl=0.5,xr=0.75,couleur=orange,style=dotted,point=2]
	\tangentetikz[liste=0/1.5/0§1/2/-0.333§2/0/-5,xl=0.33,xr=0,couleur=blue,style=densely dashed,point=3]
	...
\end{tikzpicture}
\end{minted}

\section{Exemple avec \og personnalisation \fg}

\begin{center}
\begin{tikzpicture}[x=0.5cm,y=0.5cm,xmin=0,xmax=16,xgrilles=1,ymin=0,ymax=16,ygrilles=1]
		\draw[xstep=\xgrilles,ystep=\ygrilles,line width=0.3pt,lightgray] (\xmin,\ymin) grid (\xmax,\ymax) ;
		\draw[line width=1.5pt,->,darkgray] (\xmin,0)--(\xmax,0) ;
		\draw[line width=1.5pt,->,darkgray] (0,\ymin)--(0,\ymax) ;
		\foreach \x in {0,2,...,14} {\draw[gray,line width=1.5pt] (\x,4pt) -- (\x,-4pt) ;}
		\foreach \y in {0,2,...,14} {\draw[gray,line width=1.5pt] (4pt,\y) -- (-4pt,\y) ;}
		%la liste pour la courbe d'interpolation
		\def\liste{0/6/3§3/11/0§7/3/0§10/0/0§14/14/6}
		%les tangentes "stylisées"
		\tangentetikz[liste=\liste,xl=0,xr=1,couleur=blue,style=dashed]
		\tangentetikz[liste=\liste,xl=2,xr=2,couleur=purple,style=dotted,point=2]
		\tangentetikz[liste=\liste,xl=2,xr=2,couleur=orange,style=<->,point=3]
		\tangentetikz[liste=\liste,xl=2,xr=0,couleur=ForestGreen,point=5]
		%la courbe en elle-même
		\splinetikz[liste=\liste,affpoints=true,coeffs=3,couleur=cyan,style=densely dotted]
	\end{tikzpicture}
\end{center}

\begin{minted}[frame=lines,framesep=2mm,bgcolor=LightGray,fontsize=\footnotesize,tabsize=4,breaklines]{tex}
\tikzset{%
	xmin/.store in=\xmin,xmin/.default=-5,xmin=-5,
	xmax/.store in=\xmax,xmax/.default=5,xmax=5,
	ymin/.store in=\ymin,ymin/.default=-5,ymin=-5,
	ymax/.store in=\ymax,ymax/.default=5,ymax=5,
	xgrille/.store in=\xgrille,xgrille/.default=1,xgrille=1,
	xgrilles/.store in=\xgrilles,xgrilles/.default=0.5,xgrilles=0.5,
	ygrille/.store in=\ygrille,ygrille/.default=1,ygrille=1,
	ygrilles/.store in=\ygrilles,ygrilles/.default=0.5,ygrilles=0.5,
	xunit/.store in=\xunit,unit/.default=1,xunit=1,
	yunit/.store in=\yunit,unit/.default=1,yunit=1
}

\begin{tikzpicture}[x=0.5cm,y=0.5cm,xmin=0,xmax=16,xgrilles=1,ymin=0,ymax=16,ygrilles=1]
	\draw[xstep=\xgrilles,ystep=\ygrilles,line width=0.3pt,lightgray] (\xmin,\ymin) grid (\xmax,\ymax) ;
	\draw[line width=1.5pt,->,darkgray] (\xmin,0)--(\xmax,0) ;
	\draw[line width=1.5pt,->,darkgray] (0,\ymin)--(0,\ymax) ;
	\foreach \x in {0,2,...,14} {\draw[gray,line width=1.5pt] (\x,4pt) -- (\x,-4pt) ;}
	\foreach \y in {0,2,...,14} {\draw[gray,line width=1.5pt] (4pt,\y) -- (-4pt,\y) ;}
	%la liste pour la courbe d'interpolation
	\def\liste{0/6/3§3/11/0§7/3/0§10/0/0§14/14/6}
	%les tangentes "stylisées"
	\splinetgte[liste=\liste,xl=0,xr=1,couleur=blue,style=dashed]
	\splinetgte[liste=\liste,xl=2,xr=2,couleur=purple,style=dotted,point=2]
	\splinetgte[liste=\liste,xl=2,xr=2,couleur=orange,style=<->,point=3]
	\splinetgte[liste=\liste,xl=2,xr=0,couleur=ForestGreen,point=5]
	%la courbe en elle-même
	\splinetikz[liste=\liste,affpoints=true,coeffs=3,couleur=cyan,style=densely dotted]
\end{tikzpicture}
\end{minted}

\newpage

\section{Codes du package splinetikz.sty}

\begin{minted}[frame=lines,framesep=2mm,bgcolor=LightGray,fontsize=\footnotesize,tabsize=2]{tex}
\NeedsTeXFormat{LaTeX2e}
\ProvidesPackage{splinetikz}[2022/02/10 - v1.0 - Splines cubiques, en TikZ]

%------Packages utiles
\RequirePackage[dvipsnames,table]{xcolor}
\RequirePackage{tikz}
\RequirePackage{pgf,pgffor,pgfplots}
\pgfplotsset{compat=1.18}
\RequirePackage{ifthen}
\RequirePackage{xkeyval}
\RequirePackage{xfp}
\RequirePackage{xstring}
\RequirePackage{simplekv}
\RequirePackage{listofitems}
\RequirePackage{xintexpr}
\usetikzlibrary{decorations.pathreplacing}
\usetikzlibrary{decorations.markings}
\usetikzlibrary{arrows.meta}
\end{minted}

\begin{minted}[frame=lines,framesep=2mm,bgcolor=LightGray,fontsize=\footnotesize,tabsize=2]{tex}
%------commande utile pour extraire des infos d'une liste
\newcommand\extractcoeff[2]{% #1=liste & #2=numero
	\setsepchar{§}%
	\readlist\lcoeffs{#1}
	\ifnum \lcoeffslen=1
		\def\COEFFA{#1}
		\def\COEFFB{#1}
	\else
		\itemtomacro\lcoeffs[#2]\COEFF
		\IfSubStr{\COEFF}{/}%
			{\StrCut{\COEFF}{/}{\COEFFA}{\COEFFB}}%
			{\def\COEFFA{\COEFF}\def\COEFFB{\COEFF}}
	\fi
}
\end{minted}

\begin{minted}[frame=lines,framesep=2mm,bgcolor=LightGray,fontsize=\footnotesize,tabsize=2]{tex}
%------tangente(s) en TikZ, avec point/dérivée ou liste points/dérivées
\defKV[tgte]{%
	liste=\def\TGTliste{#1},%
	width=\def\TGTepaisseur{#1},%
	couleur=\def\TGTcouleur{#1},%
	xl=\def\TGTXL{#1},%
	xr=\def\TGTXR{#1},%
	style=\def\TGTstyle{#1},%
	point=\def\TGTnumpt{#1}
}

\setKVdefault[tgte]{
	liste=,%
	width=1.25pt,%
	couleur=red,%
	xl=0.5,xr=0.5,%
	style=,%
	point=1
}

\newcommand\tangentetikz[1][]{%
	\useKVdefault[tgte]%
	\setKV[tgte]{#1}% on paramètres les nouvelles clés et on les simplifie
	\setsepchar[.]{§./}%
	\readlist\TGTlistepoints\TGTliste
	\itemtomacro\TGTlistepoints[\TGTnumpt,1]\xa
	\itemtomacro\TGTlistepoints[\TGTnumpt,2]\ya
	\itemtomacro\TGTlistepoints[\TGTnumpt,3]\fprimea
	\def\TGTDEB{\fpeval{\xa-\TGTXL}}\def\TGTFIN{\fpeval{\xa+\TGTXR}}%
	\draw[line width=\TGTepaisseur,\TGTcouleur,domain=\TGTDEB:\TGTFIN,\TGTstyle] plot (\x,{\fprimea*(\x-\xa)+\ya}) ;%
}
\end{minted}

\begin{minted}[frame=lines,framesep=2mm,bgcolor=LightGray,fontsize=\footnotesize,tabsize=2,breaklines]{tex}
%------splines en tikz avec ..controls
\defKV[spline]{%
	liste=\def\SPLliste{#1},%
	width=\def\SPLepaisseur{#1},%
	couleur=\def\SPLcouleur{#1},%
	coeffs=\def\SPLcoeffs{#1},%
	couleurpoints=\def\SPLcouleurpoints{#1},%
	taillepoints=\def\SPLtaillepoints{#1},%
	style=\def\SPLstyle{#1}
}

\setKVdefault[spline]{%
	liste=,%
	width=1.25pt,%
	couleur=red,%
	coeffs=3,%
	couleurpoints=black,%
	taillepoints=2pt,%
	affpoints=false,%
	style=
}

\newcommand\splinetikz[1][]{%
	\useKVdefault[spline]
	\setKV[spline]{#1}% on paramètres les nouvelles clés et on les simplifie
	%on lit la liste des points/nbderivés et on stocke dans \listepoints
	\setsepchar[.]{§./}
	\readlist\SPLlistepoints\SPLliste
	\def\SPLnbsplines{\inteval{\SPLlistepointslen-1}}
	%si uniquement deux points, pas de boucle...
	\ifnum \SPLlistepointslen=2
		%extraction des coeffs de compensation
		\extractcoeff{\SPLcoeffs}{1}
		%extraction des coordonnées
		\itemtomacro\SPLlistepoints[1,1]\xa
		\itemtomacro\SPLlistepoints[1,2]\ya
		\itemtomacro\SPLlistepoints[1,3]\fprimea
		\itemtomacro\SPLlistepoints[2,1]\xb
		\itemtomacro\SPLlistepoints[2,2]\yb
		\itemtomacro\SPLlistepoints[2,3]\fprimeb
		\draw[line width=\SPLepaisseur,\SPLcouleur,\SPLstyle] (\xa,\ya) ..controls +({atan \fprimea}:{(\xb-\xa)/\COEFFA}) and +({-180 + atan \fprimeb}:{(\xb-\xa)/\COEFFA}).. (\xb,\yb) ;%
	%sinon on construit bout par bout !
	\else
		\foreach \i in {1,2,...,\SPLnbsplines}{
			%extraction des coeffs de compensation
			\extractcoeff{\SPLcoeffs}{\i}
			\def\j{\inteval{\i+1}}
			\itemtomacro\SPLlistepoints[\i,1]\xa
			\itemtomacro\SPLlistepoints[\i,2]\ya
			\itemtomacro\SPLlistepoints[\i,3]\fprimea
			\itemtomacro\SPLlistepoints[\j,1]\xb
			\itemtomacro\SPLlistepoints[\j,2]\yb
			\itemtomacro\SPLlistepoints[\j,3]\fprimeb
			\draw[line width=\SPLepaisseur,\SPLcouleur,\SPLstyle] (\xa,\ya) ..controls +({atan \fprimea}:{(\xb-\xa)/\COEFFA}) and +({-180 + atan \fprimeb}:{(\xb-\xa)/\COEFFB}).. (\xb,\yb) ;%
		}
	\fi
	\ifboolKV[spline]{affpoints}%on affiche les points de contrôle
	{%
		\foreach \i in {1,2,...,\SPLlistepointslen}{%
			\itemtomacro\SPLlistepoints[\i,1]\xa
			\itemtomacro\SPLlistepoints[\i,2]\ya
			\filldraw[\SPLcouleurpoints] (\xa,\ya) circle[radius=\SPLtaillepoints] ;%
		}
	}
	{}
}
\end{minted}

\end{document}